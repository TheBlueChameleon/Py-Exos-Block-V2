\documentclass[
	english,
	fontsize=10pt,
	parskip=half,
	titlepage=true,
	DIV=12
]{scrartcl}

\usepackage[utf8]{inputenc}
\usepackage{babel}
\usepackage[T1]	{fontenc}
\usepackage{lmodern}
\usepackage{microtype}
\usepackage{color}
\usepackage{csquotes}

\usepackage{hyperref}


\usepackage{graphicx}
\usepackage{wrapfig}
\usepackage[bf]{caption}
	\captionsetup{format=plain}

\newcommand*{\tabcrlf}{\\ \hline}

\usepackage{amsmath}

\usepackage{minted}
	\usemintedstyle{friendly}

\newcommand*{\inPy}[1]{\mintinline{python3}{#1}}
\newcommand*{\ie}{i.\;e. }
\newcommand*{\eg}{e.\;g. }

\newcommand{\thus}{\ensuremath{\rightarrow}}

\begin{document}

\part*{Python Problems 07, Spring 2021}
\section{Selection Sort (1 P)}
Implement the \emph{Selection Sort Algorithm}\footnote{Of course, in \enquote{real life}, you would simply use Python's \inPy{list.sort}. This isn't only easier to use -- Python's sorting algorithm has way better runtime behaviour, \ie it's much faster.}
The algorithm is easy to implement and we repeat the ideas generally needed for coding. Follow these steps to implement the algorithm:
\begin{itemize}
\item Find the smallest element in a given list
\item Swap this smallest element with the first element of the list
\item Now look for the \emph{second smallest} element in the list, and swap it with the \emph{second} element in the list.
\item Continue like that until the list is completely sorted
\end{itemize}

Implement your solution as a function. Your solution should be applicable like that:
\mint{python}{selectionSort(unsortedList)}
where \texttt{selectionSort} is the name of your function, and \texttt{unsortedList} is a list of random numbers provided by you. After the call, \texttt{unsortedList} should hold a \emph{sorted} list.

\emph{Hints:}\\
\begin{itemize}
\item A \emph{triangle swap} is unnecessary; you can simply swap the content of two variables \texttt{a, b} by typing:
	\mint{python}{a, b = b, a}
\item Look up \texttt{help(random.sample)} to create a list of random numbers.
\item You can combine \inPy{enumerate} and \inPy{min} to find the \emph{index} of the smallest element in a list with a single line of code. To do so, use the
	technique \emph{list comprehension}\\
	Of course, there are other means of finding the index of the smallest element in a list which you may find less complicated to use.+
\end{itemize}


\section{Integral (II) (3 P)}
Write a function that takes a function as an argument, and that \emph{returns a function}. The returned function should compute an approximation for the antiderivative, \ie the integral at a given point \texttt{x}. An optional argument \texttt{N} should be usable to set the accuracy of the approximation.
In other words, your function \texttt{antiderivative} should be usable in the following way:

\begin{minted}[linenos]{python3}
import math

# def antiderivative (f, N = 1000)
#     your code here

F = antiderivative( math.sin )

print( F(math.pi) )    # should output approximately 2.0
\end{minted}

\emph{Hint}:\\
You can use most of your code from sheet 5, task 4.

\emph{Hint}:\\
As lower boundary of your integray, use the value \inPy{0}

\end{document}
